\documentclass{article}
\usepackage[margin = 1in]{geometry}
\usepackage[utf8]{inputenc}
\usepackage{graphicx}
\begin{document}
\begin{flushleft}
\noindent
Robert Hakulin\\
Stian Howard\\
Scott Baker\\
APPM 3310-002, Lyles\\
Due: April 7, 2017\\
Project Proposal\\
\end{flushleft}
\begin{center}
\textbf{\Large{NCAA Basketball Team Rankings}}
\end{center}
\section{Introduction}
Head-to-head competition lies at the root of sports. For basketball, one of the most exciting displays of this competition is the NCAA Tournament. This 68-team bracket challenges each team that gets the opportunity to play in it. While the teams are embracing the madness, mathematicians enjoy putting their prediction algorithms to the test. But before the tournament begins, a committee has the difficult task of picking the 68 teams able to compete in this tournament. But just as the basketball teams work their hardest to run the field, mathematicians work to make the most accurate algorithm to figure out which teams will get this opportunity. Despite great unpredictability in the outcomes of basketball games, prediction and ranking algorithms are continuously refined from the access to more statistics and trends in data that reveal more important attributes to a team's success. In a broader sense, being able to better predict winners, rankings, and outcomes not only makes Vegas-odds-making more interesting, but it exemplifies the sheer power of real-world analysis using mathematics.
\section{Relation to Matrix Methods}
There are many ways to attempt to rank sports teams ranging from the simplest form of solely looking at a teams win/loss record to methods that account for factors such as the strength of a team's opponent, score differential in each game, whether the game is played at home or away, etc. In each case however, a matrix representing a set of differential equations including each factor can be made to find a rating for each team ranging from 0 to 1. This matrix is called a Colley matrix if no weighting scheme is applied for factors such as score differential. It can be shown that the Colley matrix is always symmetric and positive definite which allows the Cholesky decomposition of matrices to be applied. If we are to potentially strengthen the accuracy of this rating system, a weighted matrix can be applied to account for a variety of factors. The weighting scheme will alter depending on the factors that considered. 
\section{Our Plan}
For the purpose and scope of this class, we will apply a fairly simple weighting scheme to the Colley matrix. This weighting scheme will only account for simpler factors such as score differential and time of the sports season. We believe that the score differential in each game can provide further insight on how a team performs in their games. We also believe that the time of season in which game is played factors heavily in a teams final season ranking. This is because the end of a sports season typically holds larger weight in terms of making the playoffs or end of season tournament (NCAA March Madness). We will apply this weighted matrix to a previous season for a specific division and compare our team ratings with those from a well known source such as ESPN. We hope to see that the ratings that we determine resemble those from the credible source.
\end{document}
